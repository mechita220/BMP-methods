\documentclass[]{article}
\usepackage[version=3]{mhchem}
\usepackage{sectsty}
\usepackage{amsmath}
\usepackage{hyperref}

\title {Calculation of Biochemical Methane Potential (BMP)\footnote{
  Recommended citation: 
Hafner, S.D. Calculation of Biochemical Methane Potential (BMP). Standard BMP Methods document 201, version 1.5. Available online: https://www.dbfz.de/en/BMP (accessed on Feb 28, 2020).
\newline
  Or see \url{https://www.dbfz.de/en/BMP} for a BibTeX file that can be imported into citation management software.
}}
\author{Sasha D. Hafner, Konrad Koch\\
\texttt{sasha.hafner@eng.au.dk}
}

\date{\today \\
\bigskip
\textit{
  Document number 200.
  File version 1.5. 
  This document is from the Standard BMP Methods collection.
    \footnote{For more information and other documents, visit \url{https://www.dbfz.de/en/BMP}. 
    For document version history or to propose changes, visit \url{https://github.com/sashahafner/BMP-methods}.}
}
}

\begin{document}
\maketitle

\section{Background}
This document describes calculation of biochemical methane potential (also called biomethane potential) (BMP) from laboratory measurements made in a batch BMP test.
The calculations are based on standardized \ce{CH4} volume produced in bottles with: 1) inoculum only and 2) with substrate and inoculum, along with information on the quantity of inoculum and substrate organic matter (typically as volatile solids, VS, although chemical oxygen deman, COD, is also used) added to each bottle.
For details on calculating \ce{CH4} production, see method-specific documents from the Standard BMP Methods collection\footnote{
  Available methods are: volumetric (document 201), manometric (document 202), gravimetric (document 203), and gas density (document 204), all of which can be downloaded from \url{https://www.dbfz.de/en/BMP}.
}.

\subsection{Selection of a BMP duration}
The duration at which to evaluation BMP, i.e., the length of time of the incubation, should be at least as long as the 1\% net duration\footnote{
  For a detailed description, see document 100 from \url{https://www.dbfz.de/en/BMP}.
}.
Regardless, it is important that the time is identical for both inoculum-only and inoculum + substrate bottles when carrying out calculation of BMP.
This does not mean it cannot vary among substrates from within the same BMP test--it can.
Different substrates degrade at different rates, and therefore some will require more time than others to reach the criterion.

\subsection{SMP and BMP}
The term specific methane production (SMP) is used to refer to \ce{CH4} yield from a particular substrate, in the same units as BMP\footnote{
  Standardized \ce{CH4} volume (dry, 0$^\circ$C, 101.325 kPa, referred to as ``normal'' volume) per unit mass of substrate VS added (e.g., mL g$^{-1}$, often written as NmL\textsubscript{CH\textsubscript{4}} g\textsubscript{VS}\textsuperscript{-1}).
}.
BMP is in fact simply SMP at a single suitable duration.
SMP curves, i.e., SMP values for all measurement intervals, show the development of SMP over time, reflecting kinetics of a particular test, and are commonly included in BMP reports.
The calculations described below are used for both SMP and BMP.

\section{Calculation of SMP and BMP}
These calculations require the following variables.
Units may differ, but typical units are listed below.
\begin{itemize}
  \item $V$\textsubscript{{CH$_4$, $S, i, t$}}, the standardized volume of \ce{CH4} produced in bottle $i$ containing inoculum and substrate at time $t$ (mL)
  \item $V$\textsubscript{{CH$_4$, $I, j, t$}}, the standardized volume of \ce{CH4} produced in bottle $j$ with inoculum only (``blank'' bottle) at time $t$ (mL)
  \item $m_{I, i}$, the mass of inoculum (typically as-measured (fresh) mass) originally added to bottle $i$ (g)
  \item $m_{VS, S, i}$, the mass of substrate volatile solids (VS) originally added to bottle $i$ (g)
  \item $n$, the number of replicate bottles with inoculum + substrate (typically 3, the minimum)
  \item $k$, the number of replicate bottles with inoculum only (``blanks'', typically 3, the minimum)
\end{itemize}

Methane productivity of inoculum (mL g$^{-1}$) is calculated separately for each inoculum-only bottle at each measurement time using Eq. (\ref{eq:inoc_production}).
\begin{equation}
  \label{eq:inoc_production}
  v\textsubscript{{CH$_4$, $I, j, t$}} = V\textsubscript{{CH$_4$, $I, j, t$}} / m_{I, j} 
\end{equation}
And from these, a mean value is calculated as 
\begin{equation}
  \label{eq:inoc_productivity}
  \bar{v}\textsubscript{{CH$_4$, $I, t$}} = \sum_{j = 1} ^k v\textsubscript{{CH$_4$, $I, t$}} / k
\end{equation}
where $k$ = the number of inoculum-only bottles.

Net \ce{CH4} production from inoculum + substrate bottles (mL), i.e., an estimate of \ce{CH4} production derived from substrate only\footnote{This calculation is based on the assumption of additivity for \ce{CH4} production, i.e., production of \ce{CH4} from inoculum is not affected by the presence of substrate. This is almost certainly incorrect, but similar results even when varying the inoculum-to-substrate ratio and plausible differences between measured and maximum theoretical BMP suggest it is not a large source of error.} is calculated as given in Eq. (\ref{eq:net_CH4}).
\begin{equation}
  \label{eq:net_CH4}
  V\textsubscript{CH$_4$, $S, i, t, net$} = V\textsubscript{CH$_4$, $S, i, t$} - \bar{v}\textsubscript{{CH$_4$, $I, t$}} \cdot m_{I, i}
\end{equation}
Note that the units on inoculum mass $m_{I, i}$ are completely irrelevant and have no effect of results, as long as they are sufficiently precise.
Fresh (wet) mass is recommended, for simplicity, although dry or VS mass could be used.\footnote{Any error in determination of inoculum dry matter or VS content here is exactly canceled by the combination of Eqs. (\ref{eq:inoc_production}) and (\ref{eq:net_CH4}), so has no effect. However, accurate determination of inoculum VS content is needed for calculation of inoculum-to-substrate ratio.}.
Equation (\ref{eq:net_CH4}) may be simplified if all bottles (with and without substrate) have exactly the same mass of inoculum, but because this is never completely true, Eq. (\ref{eq:net_CH4}) should always be used. 

SMP (mL g$^{-1}$) for an individual bottle at a particular duration is calculated by normalizing net \ce{CH4} production by substrate VS mass:
\begin{equation}
  \label{eq:yield}
  B_{i, t} = V\textsubscript{CH$_4$, $S, i, t, net$} / m_{VS, S, i}
\end{equation}
The mean of these values is calculated by substrate:
\begin{equation}
  \label{eq:BMP}
  \bar{B_t} = \sum_{i = 1} ^n B_{i, t} / n
\end{equation}
where $n$ is the number of replicate bottles.
Collectively all $\bar{B_t}$ values for a single substrate or all $B_{i, t}$ values for a single bottle represent an SMP curve.
BMP for a substrate is taken as $\bar{B_t}$ at an appropriate duration.

\section{Calculation of random error}
Calculation of random error in BMP estimates must include at least two sources of error: variation in \ce{CH4} production among substrate bottles, and variation in apparent inoculum \ce{CH4} productivity in blanks (inoculum-only bottles).
Uncertainty in determination of substrate VS quantity added to each bottle may sometimes be a significant source of error, and, while not required, its inclusion is recommended. 
Here, these three sources all are quantified using standard error and will be referred to as $s_{\bar{x},1}$ (substrate yield), $s_{\bar{x},2}$ (inoculum yield), and $s_{\bar{x},3}$ (substrate VS). 
Note that $s_{\bar{x},1}$ and $s_{\bar{x},2}$ may include many sources of error that collectively contribute to the observed value. 
Other sources of random error are assumed to be small: determination of inoculum and substrate mass in particular. 
Systematic error, which may be more important in some cases, is not included here. 

Units on all three standard errors are the units of the final BMP estimates, e.g., mL g$^{-1}$, standardized \ce{CH4} volume from substrate in mL per g substrate VS. 
They can be added together to provide a total estimate with:
\begin{equation}
  \label{eq:se_sum}
  s_{\bar{x},BMP} = \sqrt{\sum_{i=1} ^2 s_{\bar{x},i}^2}
\end{equation}

or, when all three sources are included:

\begin{equation}
  \label{eq:se_sum_3}
  s_{\bar{x},BMP} = \sqrt{\sum_{i=1} ^3 s_{\bar{x},i}^2}
\end{equation}
Given $s_{\bar{x},BMP}$, standard deviation $s_{BMP}$ can be calculated by multiplying by $\sqrt{n}$.

\begin{equation}
  \label{eq:sd}
  s_{BMP} = \sqrt{n} \cdot s_{\bar{x},BMP}
\end{equation}

This approach is preferred in the literature, although the use of standard error is perhaps less ambiguous\footnote{The interpretation of this standard deviation is ambiguous when the number of replicate bottles is not the same for blanks and bottles with substrate and inoculum (or when VS determination is included, when it is based on a different number of subsamples). 
In contrast, $s_{\bar{x},BMP}$ has a clear interpretation: it is an estimate of the standard deviation in the BMP $\bar{B_t}$ for a particular substrate over multiple hypothetical experiments with no change in systematic error.}. 

The value of $s_{\bar{x},1}$ is calculated from BMP values calculated for individual bottles: 
\begin{equation}
  \label{eq:se1_calc}
  s_{\bar{x},1} = \sqrt{\sum_{i=1} ^n \frac{(B_i - \bar{B})^2} {n - 1} }
\end{equation}

To calculate $s_{\bar{x},2}$ the standard error of inoculum \ce{CH4} productivity (e.g., in mL per g inoculum mass) is first calculated from individual values determined from each individual blanks. 
\begin{equation}
  \label{eq:seI_calc}
  s_{\bar{x},I} = \sqrt{\sum_{j=1} ^k \frac{(v_{CH_4, I, j} - \bar{v}_{CH_4, I})^2} {k -1} }
\end{equation}

For each individual substrate bottle an estimate of $s_{\bar{x},2}$ is then made as:
\begin{equation}
  \label{eq:se2i_calc}
  s_{\bar{x},2,i} = s_{\bar{x},I} \cdot m_{I, i} 
\end{equation}

Equation (\ref{eq:se2i_calc}) includes an assumption that error in inoculum mass determination is negligible compared to error in inoculum \ce{CH4} yield, which is reasonable if inoculum quantity is determined by mass. 
Given these values, $s_{\bar{x},2}$ is then calculated by substrate with:
\begin{equation}
  \label{eq:se2_calc}
  s_{\bar{x},2} = \sqrt{\frac{\sum_{i=1} ^n s_{\bar{x},2,i}^2} {n}}
\end{equation}

The third (and optional) source of random error ($s_{\bar{x},3}$) is determined from the relative standard error in substrate VS content determination $rs_{\bar{x},VS}$ (dimensionless, expressed as a fraction of VS content (e.g., \% of fresh mass)).
This approach is based on the assumption that the primary source of error in substrate VS mass comes from measurement of substrate VS content. 
Given this value, an estimate of $s_{\bar{x},3}$ can be made for each bottle with:
\begin{equation}
  \label{eq:se3i_calc}
  s_{\bar{x},3,i} = rs_{\bar{x},VS} \cdot B_{i}
\end{equation}

Equation (\ref{eq:se3i_calc}) differs from Eq. (\ref{eq:se2i_calc}) by inclusion of $B_{i}$ because for $s_{\bar{x},3}$ error is propagated through division and not subtraction. 
Given these values for individual bottles, $s_{\bar{x},3}$ is calculated from Eq. (\ref{eq:se3_calc}).
\begin{equation}
  \label{eq:se3_calc}
  s_{\bar{x},3} = \sqrt{\frac{\sum_{i=1} ^n s_{\bar{x},3,i}^2} {n}}
\end{equation}



\end{document}

\documentclass[]{article}
\usepackage[version=3]{mhchem}
\usepackage{natbib}
\bibliographystyle{abbrvnat}
\usepackage{amsmath}
\usepackage[colorlinks]{hyperref}

\hypersetup{
  citecolor = {blue},
}

\newcommand{\unit}[1]{\ensuremath{\, \mathrm{#1}}}


\title {Calculation of Methane Production from Manometric Measurements\footnote{
  Recommended citation: 
Hafner, S.D.; Astals, S.; Buffiere, P.; L\o jborg, N.; Holliger, C.; Koch, K.;, Weinrich, S., Calculation of Methane Production from Manometric Measurements. Standard BMP Methods document 201, version 2.6. Available online: https://www.dbfz.de/en/BMP (accessed on April 19, 2020).
\newline
  Or see \url{https://www.dbfz.de/en/BMP} for a BibTeX file that can be imported into citation management software.
}}
\author{Sasha D. Hafner, Sergi Astals, \\ Pierre Buffiere, Nanna L{\o}jborg, \\ Christof Holliger, Konrad Koch, \\ and S{\"o}ren Weinrich\\
\\
\texttt{sasha.hafner@eng.au.dk} (S. D. Hafner)\\
\texttt{sastals@ub.edu} (S. Astals)\\
} 

\date{\today \\
\bigskip
\textit{
  Document number 202.
  File version 2.6. 
  This document is from the Standard BMP Methods collection.
    \footnote{For more information and other documents, visit \url{https://www.dbfz.de/en/BMP}. 
    For document version history or to propose changes, visit \url{https://github.com/sashahafner/BMP-methods}.}
}
}

\begin{document}
\maketitle

\section{Introduction}
This document describes calculations for manometric measurment of biogas, typically from laboratory experiments aimed at measuring biochemical methane potential (BMP).
Manometric measurement systems are not described in detail here, but descriptions can be found elsewhere \citep{rozziMethodsAssessingMicrobial2004,vdiFermentationOrganicMaterials2016}.

In manometric methods (as in some volumetric methods), biogas is typically allowed to accumulate within the headspace of a sealed bottle over \textit{measurement intervals}, which are the periods between measurements.
The pressure and composition of the bottle headspace are measured at the end of each measurement interval.
Measurements might be \textit{interval}, where the bottle headspace is completely vented to ambient pressure after each measurement.
Or measurements may be \textit{cumulative}, where bottles are not vented until the end of the test.
A combination is also possible.
Once measurements are converted to standardized biogas and \ce{CH4} volume, cumulative data can be calculated from interval or vice versa.

As with volumetric methods, two methods are commonly used to address the problem of biogas dilution with flushing gas, and both are described here: one based on normalized \ce{CH4} concentrations (method 1) and one that explicitly includes estimation of \ce{CH4} in the bottle headspace (method 2).
Expected results from the two methods are effectively identical; differences in cumulative \ce{CH4} production are due only to error in measurement of biogas composition or headspace volume, in addition to small effects of changes in biogas composition between measurement intervals.

\section{Standardization of measured gas volume}
The approach used for standardization of biogas volume is identical for all biogas methods \citep{BMPdoc202man, BMPdoc204gasdens, BMPdoc201vol, BMPdoc203grav}.
In manometric methods, measured volume is typically fixed at the volume of bottle headspace, and instead pressure varies.
Standard gas volume is calculated from measured volume ($V_{meas}$) by correcting for water vapor, temperature, and pressure:
\begin{equation}
  \label{eq:stdvol}
  V_{std} = V_{meas} \cdot \frac{(p_{meas} - p\textsubscript{H$_2$O})} {101.325 \unit{kPa}} \cdot \frac {273.15 \unit{K}}{(T_{meas} + 273.15)}
\end{equation}
where $p_{meas}$ is the gas pressure (kPa), $T_{meas}$ is the gas temperature at the time of volume measurement in $^\circ$C, $p\textsubscript{H$_2$O}$ the water vapor partial pressure (kPa), 273.15 K (0$^\circ$C) is the standard temperature, 101.325 kPa is the standard pressure (kPa), and $V_{std}$ is the standardized gas volume.
Eq. (\ref{eq:stdvol}) combines Boyle's, Charles's, and Dalton's laws \citep{negiTextbookPhysicalChemistry1985}, which are appropriate for the pressure and temperature ranges encountered in BMP experiments.
%Note that Eq. (\ref{eq:stdvol}) is not exactly the same as the ideal gas law, which includes the ideal gas constant, while Eq. (\ref{eq:stdvol}) includes no assumption about molar volume.
Other units can, of course, be used, but standard temperature and pressure must be equivalent (e.g., 1.01325 bar, 1.0 atm, 32$^\circ$F) and Eq. (\ref{eq:stdvol}) is based on absolute temperature (note the conversion of $^\circ$C within the equation by $+~273.15$)\footnote{
  Other standard conditions for temperature and pressure exist, but these are the values used in the biogas field.
  In particular, that this standard pressure used in the biogas field is different from the value recommended by IUPAC (1.0 bar).
}.

Typically, the value of $p\textsubscript{H$_2$O}$ for biogas should be assumed to be the saturation vapor pressure, because biogas is expected to be near equilibrium with an aqueous phase.
Saturation vapor pressure can be calculated using, e.g., the Magnus-form equation given below (Eq. 21 in \citet{alduchovImprovedMagnusForm1996})\footnote{
  Other options exist \citep{richardsMethodsKineticanalysisMethane1991, vdiFermentationOrganicMaterials2016}, and will provide nearly identical values.
}.

\begin{equation}
\label{eq:magnus}
   p\textsubscript{H$_2$O} = 0.61094 e^{(17.625 \cdot T_{meas}/(243.04 + T_{meas}))}
\end{equation}

Immediately after venting, when water vapor has been lost through venting but water evaporation has not yet led to equilibrium, it may be reasonable to assume that the mixing ratio of water is the same as it was prior to venting.
Assuming saturation prior to venting and an insignificant change in temperature, this is equivalent to taking relative humidity as the ratio of post- to pre-venting absolute pressure (always $<1.0$). 
Alternatively, it may be no less accurate to assume 100\% relative humidity after venting as well, especially if more than a few minutes pass between venting and measurement of residual pressure.
Althought the difference in resulting BMP estimates is not large (perhaps a few \%), it is prudent to state which of these two assumptions was made when reporting BMP results.

For interval-based measurements, biogas production in interval $i$ is calculated as:
\begin{equation}
  \label{eq:bgint}
  V_{b, std, i} = V_{h, std, pre, i} - V_{h, std, post, i - 1}
\end{equation}
where all $V_{b, std}$ is standardized biogas gas volume produced in an interval, $V_{h, std}$ is the standardized volume of gas within a bottle's headspace,  $pre$ and $post$ refer to before and after venting, respectively, and $i$ indicates the current interval and $i-1$ the previous one.

For a cumulative approach, the equation is similar.
Cumulative biogas production in interval $i$ is calculated as:
\begin{equation}
  \label{eq:bgcum}
  V_{b, std, i} = V_{h, std, i} - V_{h, std, start}
\end{equation}
where all $V_{h, std, setup}$ is the standardized volume of gas within the bottle headspace at the start of the test.

\section{Calculation of \ce{CH4} production}
\subsection{Method 1}
In the first method, biogas is assumed to consist of only \ce{CH4} and \ce{CO2} at the time of production (i.e., as produced by the microbial community) and \ce{CH4} production is calculated from vented (removed) biogas only.
With this method, each measurement interval is independent of the others; \ce{CH4} production for each interval is determined from the biogas volume and composition for that interval.
This method is described in \citet[Section 3]{richardsMethodsKineticanalysisMethane1991} and \citet[Eq. (7)]{vdiFermentationOrganicMaterials2016}.
Relying on the assumptions that all gas production is biogas and all \ce{CH4} and \ce{CO2} are from biogas\footnote{
  Therefore, this approach is not suitable when \ce{CO2} is included in the flushing gas.
}, this provides the simplest approach for calculating \ce{CH4} production.

Concentrations of \ce{CH4} and \ce{CO2} are normalized so they sum to 1.0:
\begin{equation}
  x\textsubscript{CH$_4$$, n$} = x\textsubscript{CH$_4$}/(x\textsubscript{CH$_4$} + x\textsubscript{CO$_2$})
\end{equation}
where $x\textsubscript{CH$_4$}$ and $x\textsubscript{CO$_2$}$ are the measured \ce{CH4} and \ce{CO2} concentrations as volume (mole) fraction (possibly including a correction for water vapor--this has no effect here) and $x\textsubscript{CH$_4$$, n$}$ is the normalized \ce{CH4} volume fraction.

Methane production in an individual measurement interval is then calculated from the standardized biogas volume measured in that interval ($V_{b,std}$) with:
\begin{equation}
  V\textsubscript{CH$_4$} = x\textsubscript{CH$_4$,$ n$} \cdot V_{b, std}
\end{equation}

If measurements are interval, cumulative \ce{CH4} production is taken as the cumulative sum of interval values.

\subsection{Method 2}
Method 2 relies on fewer assumptions, but requires the true concentration of \ce{CH4} (volume fraction) within the bottle headspace, with correction only for water vapor.
For interval measurements, \ce{CH4} production in an interval is determined from the change in bottle \ce{CH4} content during an incubation interval.
\begin{equation}
  \label{eq:ch4m2i}
  V\textsubscript{CH$_4$, i} = x\textsubscript{CH$_4$, i} \cdot V_{h, std, pre, i} - x\textsubscript{CH$_4$, i - 1} \cdot V_{h, std, post, i - 1}
\end{equation}
Cumulative production is taken as the cumulative sum of interval values. 

If measurements are cumulative (i.e., the bottle was not vented prior to interval $i$) the equation is even simpler:
\begin{equation}
  \label{eq:ch4m2c}
  V\textsubscript{CH$_4$, i} = x\textsubscript{CH$_4$, i} \cdot V_{h, std, i}
\end{equation}

\bibliography{bib}

\end{document}

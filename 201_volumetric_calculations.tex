\documentclass[]{article}
\usepackage[version=3]{mhchem}
\usepackage{natbib}
\bibliographystyle{abbrvnat}
\usepackage{amsmath}
\usepackage[colorlinks]{hyperref}

\newcommand{\unit}[1]{\ensuremath{\, \mathrm{#1}}}

\title {Calculation of Methane Production from Volumetric Measurements\footnote{
  Recommended citation: 
Hafner, S.D.; L\o jborg, N.; Holliger, C.; Koch, K.;, Weinrich, S., Calculation of Methane Production from Volumetric Measurements. Standard BMP Methods document 201, version 1.7. Available online: https://www.dbfz.de/en/BMP (accessed on April 21, 2020).
\newline
  Or see \url{https://www.dbfz.de/en/BMP} for a BibTeX file that can be imported into citation management software.
}}
\author{Sasha D. Hafner, Nanna L\o jborg, \\ Sergi Astals, Christof Holliger, \\ Konrad Koch, and S{\"o}ren Weinrich\\
\\
\texttt{sasha.hafner@eng.au.dk}
} 

\date{\today \\
\bigskip
\textit{
  Document number 201.
  File version 1.7. 
  This document is from the Standard BMP Methods collection.
    \footnote{For more information and other documents, visit \url{https://www.dbfz.de/en/BMP}. 
    For document version history or to propose changes, visit \url{https://github.com/sashahafner/BMP-methods}.}
}
}


\begin{document}
\maketitle

\section{Introduction}
This document describes calculations for volumetric measurment of biogas.
As with manometric methods, two methods are commonly used and both are described here: one based on normalized \ce{CH4} concentrations (method 1) and one that explicitly includes estimation of \ce{CH4} in the bottle headspace (method 2).
Expected results from the two methods are identical; differences are due only to error in measurement of biogas composition or headspace volume.
Both methods are available through the \texttt{cumBg()} function in the biogas package \citep{hafnerSoftwareBiogasResearch2018} and through the web application OBA (\url{https://biotransformers.shinyapps.io/oba1/}) and can be easily added to, e.g., a spreadsheet template.

\section{Standardization of measured gas volume}
Both methods use the same approach for standardization of gas volume.
Dry biogas volume in a bottle's headspace before and after venting is calculated by correcting for water vapor, temperature, and pressure.
First the measured gas volume (e.g., in a syringe or hanging water column) is converted to dry conditions at standard pressure:

\begin{equation}
  \label{eq:h2ocorrection}
  V_{dry} = V_{headspace}(P_{meas} - P_{H_2O})/101.325 \unit{kPa}
\end{equation}
where $P_{meas}$ is the measured headspace pressure and $P_{H_2O}$ the water vapor partial pressure (both in kPa).
Eq. (\ref{eq:h2ocorrection}) is an expression of Boyle's law.
The value of $P_{H_2O}$ is assumed to be the saturation vapor pressure, and can be calculated using, e.g., the Magnus-form equation given below (Eq. 21 in \citet{alduchovImprovedMagnusForm1996}):

\begin{equation}
\label{eq:magnus}
   P_{H_2O} = 0.61094 e^{(17.625 T/(243.04 + T))}
\end{equation}
where T is temperature in $^\circ$C.
Volume is then further standardized to 273.15 K by application of Charles's law:

\begin{equation}
  \label{eq:bgstd}
  V_{std} = V_{dry} 273.15 \unit{K}/T_{meas}
\end{equation}
where $V_{std}$ is the standardized volume of gas within a bottle's headspace at the time of pressure measurement.
Interval biogas production $V_{biogas, i}$ is taken as this standardized volume $v_{std}$.
Cumulative production is taken as the cumulative sum of interval values. 

\section{Calculation of methane production}
\subsection{Method 1}
In the first method, biogas is assumed to consist of only \ce{CH4} and \ce{CO2} at the time of production (i.e., as produced by the microbial community) and \ce{CH4} production is calculated from vented (removed) biogas only.
This method is described in \cite{richardsMethodsKineticanalysisMethane1991}.
Coupled with the assumption that all gas production is biogas, this provides the simplest approach for calculating \ce{CH4} production.

First, concentrations of \ce{CH4} and \ce{CO2} are adjusted so they sum to 1.0:
\begin{equation}
  x_{CH_4, n} = x_{CH_4}/(x_{CH_4} + x_{CO_2})
\end{equation}
where $x_{CH_4}$ and $x_{CO_2}$ are the measured \ce{CH4} and \ce{CO2} concentrations as volume (mole) fraction (possibly including a correction for water vapor--this has no effect here) and $x_{CH_4, n}$ is the normalized \ce{CH4} volume fraction.

Methane production in an interval $i$ is then calculated as
\begin{equation}
  V_{CH_4, i} = x_{CH_4, n} V_{biogas, i}
\end{equation}
Cumulative production is taken as the cumulative sum of interval values. 

\subsection{Method 2}
Method 2 relies on fewer assumptions, but requires the true concentration of \ce{CH4} (volume fraction) of \ce{CH4} within the bottle headspace, with correction only for water vapor.
Here, \ce{CH4} production in an interval has two components: a vented part that is naturally interval, and a residual headspace part, that is naturally cumulative:
\begin{equation}
  V_{CH_4, i} = V_{CH_4, v, i} + ( V_{CH_4, HSR, i} - V_{CH_4, HSR, i-1} )
\end{equation}
where the subscript $v$ indicates vented volume and $HSR$ = residual headspace volume (post-venting).

Vented \ce{CH4} is calculated from:
\begin{equation}
  V_{CH_4, v, i} = x_{CH_4, i} V_{biogas, i}
\end{equation}

Headspace \ce{CH4} is calculated from:
\begin{equation}
  V_{CH_4, HSR, i} = x_{CH_4, i} V_{post, i}
\end{equation}
where $V_{post}$ is the post-venting standardized volume of gas in the bottle headspace.
Cumulative production is taken as the cumulative sum of interval values. 

\section{Example Calculations}
In the following example, \ce{CH4} production is calculated from a single interval measurement made on a single bottle in a BMP trial. Calculations are made using both volumetric method 1 and 2. 
For both methods standardized gas volume is calculated from Eq. (\ref{eq:bgstd}) by correcting for water vapor, temperature, and pressure.

Measured biogas volume ($V_{biogas,i}$) was 73.6 mL at a temperature ($T_{meas}$) of 30$^\circ$C and a pressure ($P_{meas}$) of 101.325 kPa .
Measured biogas composition $x_{CH_4}$ for the given interval was 0.656, and $x_{CO_2}$ was 0.289. For the previous interval, $x_{CH_4}$ was 0.587.

First water vapor pressure is calculated at the measured headspace temperature using Eq (\ref{eq:magnus}).
\begin{equation*}
   P_{H_2O} = 0.61094 \cdot e^{\frac{17.625 \cdot 30^\circ C}{243.04 + 30^\circ C}} = 4.237\unit{kPa}
\end{equation*}
Secondly, the measured volume is converted to dry conditions at standard pressure using Eq. (\ref{eq:h2ocorrection}).
\begin{equation*}
   V_{dry} = \frac{73.6\unit{mL} \cdot (101.325\unit{kPa} - 4.237\unit{kPa})}{101.325\unit{kPa}} = 124.8\unit{mL}  
\end{equation*}
Then, volume is further standardized following Eq. (\ref{eq:bgstd}).
\begin{equation*}
    V_{std} = \frac{124.8\unit{mL} \cdot 273.15\unit{K}}{303.15\unit{K}} = 112.4\unit{mL}  
\end{equation*}
$V_{biogas,i}$ is taken as this $V_{std}$. Cumulative production is taken as the cumulative sum of interval values.

For method 2 additional calculation of standardized gas volume post venting is required in order to determine interval biogas production. 
Post-venting pressure in the current ($P_{post_{i}}$) and previous ($P_{post_{i-1}}$) intervals was assumed to be constant atmospheric pressure of 101.325 kPa. 
Bottle headspace volume was 81.3 mL. 
\begin{equation*}
   V_{dry,post,i-1} = V_{dry,post,i} = \frac{81.3\unit{mL} \cdot (101.325\unit{kPa} - 4.237\unit{kPa})}{101.325\unit{kPa}} = 77.88\unit{mL}  
\end{equation*}
Then, following Eq. (\ref{eq:bgstd}) post-venting standardized gas volume is calculated.
\begin{equation*}
    V_{post,i} = \frac{77.88\unit{mL} \cdot 273.15\unit{K}}{303.15\unit{K}} = 70.17\unit{mL}  
\end{equation*}

\subsection{Method 1}
The mole fraction of $CH_{4}$ ($x_{CH_4}$, dimensionless) normalized for $CH_{4}$ and $CO_{2}$ can be calculated according Eq. (4).
\begin{equation*}
    x_{CH{4},n} = \frac{0.656}{0.656 + 0.289} = 0.694
\end{equation*}

Then, following Eq. (5), $CH_{4}$ production in the interval is calculated.
\begin{equation*}
  V_{CH_4, i} = 0.694 \cdot 112.4\unit{mL}  = 78.04\unit{mL} 
\end{equation*}

\subsection{Method 2}
$V_{dry,post,i} =CH_{4}$ production in the interval is calculated following Eq. (6) using the true concentration of $CH_{4}$ within the bottle headspace.
First vented $CH_{4}$ volume ($V_{CH_4, v,i}$) is obtained from the interval biogas production and the mole fraction of $CH_{4}$ using Eq. (7).
\begin{equation*}
  V_{CH_4, v, i} = 0.656 \cdot 112.4\unit{mL}  = 73.74\unit{mL} 
\end{equation*}
Secondly, post-venting residual headspace pressure in the current and previous interval are calculated following Eq. (8).
\begin{equation*}
  V_{CH_4, HSR, i} = 0.656 \cdot 70.17\unit{mL}  = 46.03\unit{mL} 
\end{equation*}
\begin{equation*}
  V_{CH_4, HSR, i-1} = 0.587 \cdot 70.17\unit{mL}  = 41.19\unit{mL} 
\end{equation*}
Then, following Eq. (6), $CH_{4}$ production in the interval is calculated.
\begin{equation*}
  V_{CH_{4},i} = 73.74\unit{mL} + (46.033\unit{mL} - 41.19\unit{mL})  = 78.59\unit{mL} 
\end{equation*}

\bibliography{bib}

\end{document}

\documentclass[]{article}
\usepackage[version=3]{mhchem}
\usepackage{natbib}
\bibliographystyle{abbrvnat}
\usepackage{amsmath}
\usepackage[colorlinks]{hyperref}

\hypersetup{
  citecolor = {blue},
}

\newcommand{\unit}[1]{\ensuremath{\, \mathrm{#1}}}

\title {Calculation of Methane Production from Volumetric Measurements\footnote{
  Recommended citation: 
Hafner, S.D.; L{\o}jborg, N.; Holliger, C.; Koch, K.;, Weinrich, S., Calculation of Methane Production from Volumetric Measurements. Standard BMP Methods document 201, version 1.9. Available online: https://www.dbfz.de/en/BMP (accessed on April 21, 2020).
\newline
  Or see \url{https://www.dbfz.de/en/BMP} for a BibTeX file that can be imported into citation management software.
}}
\author{Sasha D. Hafner, Nanna L{\o}jborg, \\ Sergi Astals, Christof Holliger, \\ Konrad Koch, and S{\"o}ren Weinrich\\
\\
\texttt{sasha.hafner@eng.au.dk}
} 

\date{\today \\
\bigskip
\textit{
  Document number 201.
  File version 1.9. 
  This document is from the Standard BMP Methods collection.
    \footnote{For more information and other documents, visit \url{https://www.dbfz.de/en/BMP}. 
    For document version history or to propose changes, visit \url{https://github.com/sashahafner/BMP-methods}.}
}
}


\begin{document}
\maketitle

\section{Introduction}
This document describes calculations for processing volumetric measurements of biogas production, typically from laboratory experiments aimed at measuring biochemical methane potential (BMP).
Volumetric measurement systems are not described in detail here, but descriptions can be found elsewhere \citep{owenBioassayMonitoringBiochemical1979,rozziMethodsAssessingMicrobial2004,vdiFermentationOrganicMaterials2016}.

In volumetric methods (as in manometric methods), biogas is typically allowed to accumulate over \textit{measurement intervals}, which are the periods between measurements.
Biogas volume and composition are measured at the end of each measurement interval.
Measurements might be \textit{interval}, where the measurement system is reset after each measurement.
For example, when bottles sealed with septa are used, the biogas that accumulates in each bottle's headspace during a measurement interval is typically removed (vented) at the end of the interval, and its volume is measured. 
In the process the headspace pressure is reset to ambient.
Or measurements may be \textit{cumulative}, where the biogas quantity measured at a particular time is the cumulative total produced in all preceding measurement intervals.
For example, biogas might accumulate in a separate vessel also used to measure volume (e.g., a eudiometer) that is emptied of biogas only at the end of the experiment, although cumulative volume is recorded at the end of each measurement interval.
Measurement of biogas composition must reflect the approach used for volume, i.e., it does not make sense to work with interval volume data if composition measurements are made on a cumulative mixture of biogas produced.
Once measurements are converted to standardized biogas and \ce{CH4} volume, cumulative data can be calculated from interval or vice versa.

As with manometric methods, two methods are commonly used to address the problem of biogas dilution with flushing gas, and both are described here: one based on normalized \ce{CH4} concentrations (method 1) and one that explicitly includes estimation of \ce{CH4} in the bottle headspace (method 2).
Expected results from the two methods are effectively identical; differences in cumulative \ce{CH4} production are due only to error in measurement of biogas composition or headspace volume, in addition to small effects of changes in biogas composition between measurement intervals.

\section{Standardization of measured gas volume}
The approach used for standardization of biogas volume is identical for all biogas methods \citep{BMPdoc202man, BMPdoc204gasdens, BMPdoc201vol, BMPdoc203grav}.
Standard gas volume is calculated from measured volume ($V_{meas}$) by correcting for water vapor, temperature, and pressure:
\begin{equation}
  \label{eq:stdvol}
  V_{std} = V_{meas} \cdot \frac{(p_{meas} - p\textsubscript{H$_2$O})} {101.325 \unit{kPa}} \cdot \frac {273.15 \unit{K}}{(T_{meas} + 273.15)}
\end{equation}
where $p_{meas}$ is the gas pressure (kPa), $T_{meas}$ is the gas temperature at the time of volume measurement in $^\circ$C, $p\textsubscript{H$_2$O}$ the water vapor partial pressure (kPa), 273.15 K (0$^\circ$C) is the standard temperature, 101.325 kPa is the standard pressure (kPa), and $V_{std}$ is the standardized gas volume.
Eq. (\ref{eq:stdvol}) combines Boyle's, Charles's, and Dalton's laws \citep{negiTextbookPhysicalChemistry1985}, which are appropriate for the pressure and temperature ranges encountered in BMP experiments.
%Note that Eq. (\ref{eq:stdvol}) is not exactly the same as the ideal gas law, which includes the ideal gas constant, while Eq. (\ref{eq:stdvol}) includes no assumption about molar volume.
Other units can, of course, be used, but standard temperature and pressure must be equivalent (e.g., 1.01325 bar, 1.0 atm, 32$^\circ$F) and Eq. (\ref{eq:stdvol}) is based on absolute temperature (note the conversion of $^\circ$C within the equation by $+~273.15$)\footnote{
  Other standard conditions for temperature and pressure exist, but these are the values used in the biogas field.
  In particular, that this standard pressure used in the biogas field is different from the value recommended by IUPAC (1.0 bar).
}.

Typically, the value of $p\textsubscript{H$_2$O}$ for biogas should be assumed to be the saturation vapor pressure, because biogas is expected to be near equilibrium with an aqueous phase.
This is reasonable as long as volume is measured at or below the temperature of the bottle, and has not undergone substantial expansion due to a drop in pressure after removal from the bottle.
Or, if expansion has occured, the approach is still suitable if the gas has also been cooled enough so the temperature is at or below the dew point\footnote{
  For example, biogas within a bottle headspace with an absolute pressure of 250 kPa at 35$^\circ$C has a saturation vapor pressure for water of 5.6 kPa based on Eq (\ref{eq:stdvol}).
  If volume is then measured at ambient pressure and room temperature (ca. 101 kPa and 20 $^\circ$C) $p\textsubscript{H$_2$O}$ would be 2.4 kPa due to volume expansion.
  In comparison, saturation vapor pressure at 20 $^\circ$C is slightly lower, at 2.3 kPa, so the assumption of saturation with water vapor is reasonable.
  For a much higher headspace pressure, or higher gas temperature at the time of measurement, this assumption is not valid and $p\textsubscript{H$_2$O}$ may be adjusted based on volume expansion if the error is more than ca. 0.5\%.
}.
Saturation vapor pressure can be calculated using, e.g., the Magnus-form equation given below (Eq. 21 in \citet{alduchovImprovedMagnusForm1996})\footnote{
  Other options exist \citep{richardsMethodsKineticanalysisMethane1991, vdiFermentationOrganicMaterials2016}, and will provide nearly identical values.
}.

\begin{equation}
\label{eq:magnus}
   p\textsubscript{H$_2$O} = 0.61094 e^{(17.625 \cdot T_{meas}/(243.04 + T_{meas}))}
\end{equation}

\section{Calculation of methane production}

\subsection{Method 1}
In the first method, biogas is assumed to consist of only \ce{CH4} and \ce{CO2} at the time of production (i.e., as produced by the microbial community) and \ce{CH4} production is calculated from vented (removed) biogas only.
With this method, each measurement interval is independent of the others; \ce{CH4} production for each interval is determined from the biogas volume and composition for that interval.
This method is described in \citet[Section 3]{richardsMethodsKineticanalysisMethane1991} and \citet[Eq. (7)]{vdiFermentationOrganicMaterials2016}.
Relying on the assumptions that all gas production is biogas and all \ce{CH4} and \ce{CO2} are from biogas\footnote{
  Therefore, this approach is not suitable when \ce{CO2} is included in the flushing gas.
}, this provides the simplest approach for calculating \ce{CH4} production.

Concentrations of \ce{CH4} and \ce{CO2} are normalized so they sum to 1.0:
\begin{equation}
  x\textsubscript{CH$_4$$, n$} = x\textsubscript{CH$_4$}/(x\textsubscript{CH$_4$} + x\textsubscript{CO$_2$})
\end{equation}
where $x\textsubscript{CH$_4$}$ and $x\textsubscript{CO$_2$}$ are the measured \ce{CH4} and \ce{CO2} concentrations as volume (mole) fraction (possibly including a correction for water vapor--this has no effect here) and $x\textsubscript{CH$_4$$, n$}$ is the normalized \ce{CH4} volume fraction.

Methane production in an individual measurement interval is then calculated from the standardized biogas volume measured in that interval ($V_{b,std}$) with:
\begin{equation}
  V\textsubscript{CH$_4$} = x\textsubscript{CH$_4$,$ n$} \cdot V_{b, std}
\end{equation}

If measurements are interval, cumulative \ce{CH4} production is taken as the cumulative sum of interval values.

\subsection{Method 2}
Method 2 relies on fewer assumptions, but requires the true concentration of \ce{CH4} (volume fraction) within the bottle headspace, with correction only for water vapor.
This method was briefly described in \citet[p 488]{owenBioassayMonitoringBiochemical1979}.
Here, \ce{CH4} production in an interval has two components: a vented part that is naturally interval (but could be measured in a cumulative manner via accumulation in an external vessel), and a residual headspace part, that is naturally cumulative.
Because determination of \ce{CH4} production in an individual interval depends on multiple intervals, a subscript $i$ for measurement interval is introduced here.

Vented \ce{CH4} for interval $i$ (V\textsubscript{CH$_4, v, i$}) is calculated simply as the product of standardized vented (removed) biogas volume and \ce{CH4} mole fraction for that interval:
\begin{equation}
  V\textsubscript{CH$_4$$,v, i$} = x\textsubscript{CH$_4$,$ i$} \cdot V_{b, std, i}
\end{equation}

Cumulative \ce{CH4} production at the end of interval $i$ can be calculated by adding headspace \ce{CH4} to the cumulative sum of vented \ce{CH4}:
\begin{equation}
  V\textsubscript{CH$_4$$, cum, i$} = \sum_{j = 1}^i {V\textsubscript{CH$_4$$, v, i$}} +  x\textsubscript{CH$_4$,$ i$} \cdot V_{h, std, i}
\end{equation}
where $V_{h, std, i}$ is the post-venting total standardized volume of gas in the bottle headspace (calculated using Eq. (\ref{eq:stdvol})).
If vented measurements are cumulative instead of interval, the summation operation is omitted and a single value from interval $i$ is added instead.

\bibliography{bib}

\end{document}

\documentclass[]{article}
\usepackage[version=3]{mhchem}
\usepackage{natbib}
\bibliographystyle{abbrvnat}
\usepackage[colorlinks]{hyperref}
\hypersetup{
  citecolor = {blue},
}
\newcommand{\unit}[1]{\ensuremath{\, \mathrm{#1}}}

\title {Validation criteria for measurement of biochemical methane potential (BMP)\footnote{
  Recommended citation: 
Hafner, S.D.; Holliger, C.; Fruteau de Laclos, H.; Koch, K. Validation criteria for measurement of biochemical methane potential (BMP). Standard BMP Methods document 101, version 1.0. Available online: https://www.dbfz.de/en/BMP (accessed on August 1, 2020).
\newline
  Or see \url{https://www.dbfz.de/en/BMP} for a BibTeX file that can be imported into citation management software.
}}
\author{
Sasha D. Hafner,
Christof Holliger,
H{\'e}l{\`e}ne Fruteau de Laclos,
Konrad Koch
} 

\date{\today \\
\bigskip
\textit{
  Document number 101.
  File version 1.0. 
  This document is from the Standard BMP Methods collection.
    \footnote{For more information and other documents, visit \url{https://www.dbfz.de/en/BMP}. 
    For document version history or to propose changes, visit \url{https://github.com/sashahafner/BMP-methods}.}
}
}

\begin{document}
\maketitle

\section{Validation criteria}
\label{sec:crit}
BMP results that meet \textit{all} the following criteria can be described as ``validated''.
Otherwise tests should be repeated.

\begin{enumerate}
  \item All required components of the BMP measurement protocol described in \citet{BMPdoc100req} (document 100) are met. 
    This includes a duration criterion: Terminate BMP tests only after daily \ce{CH4} production from individual batches (bottles) during 3 consecutive days is $<$ 1.0\% of the net accumulated volume of methane from the substrate (substrate batch minus average of blanks). 
  \item Mean cellulose BMP is between 340 and 395 NmL\textsubscript{CH\textsubscript{4}} g\textsubscript{VS}\textsuperscript{-1}.
  \item Relative standard deviation for cellulose BMP (standard deviation, including variability in blanks, substrate bottles, and added substrate VS, divided by mean BMP) is no more than 6\%.
\end{enumerate}


\section{More details}
This document presents validation criteria for measurement of biochemical methane potential (also called biomethane potential) (BMP) in batch tests.
The development of these criteria is described in \citet{hafnerImprovingInterlaboratoryReproducibility2020} and they are intended to be applied after measurement of BMP in order to determine if results are valid, or if the test must be repeated.
Only results from BMP tests that have followed the requirements given in \citet{BMPdoc100req} (document 100) can be considered ``validated''.
The criteria listed above are duplicated in this document, and the present document was created to simply make it easier to find these required criteria.
For details and many additional recommendations, see \citet{holligerStandardizationBiomethanePotential2016}, \citet{hafnerImprovingInterlaboratoryReproducibility2020}, and \citet{BMPdoc100req}.

\bibliography{bib}

\end{document}

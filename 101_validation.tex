\documentclass[]{article}
\usepackage[version=3]{mhchem}
\usepackage{natbib}
\bibliographystyle{abbrvnat}
\usepackage[colorlinks]{hyperref}
\hypersetup{
  citecolor = {blue},
}
\newcommand{\unit}[1]{\ensuremath{\, \mathrm{#1}}}

\title {A summary of requirements for measurement and validation of biochemical methane potential (BMP)\footnote{
  Recommended citation (for document 100): 
Holliger, C.; Fruteau de Laclos, H.; Hafner, S.D.; Koch, K.; Weinrich, S.; Astals, S.; Alves, M.; Andrade, D.; Angelidaki, I.; Appels, L.; Astals, S.; Azman, S.; et al. Requirements for measurement and validation of biochemical methane potential (BMP). Standard BMP Methods document 100, version 1.6. Available online: https://www.dbfz.de/en/BMP (accessed on September 7, 2020).
\newline
  Or see \url{https://www.dbfz.de/en/BMP} for a BibTeX file.
}\vspace{-4ex}}

\author{
} 

\date{\today \\
\bigskip
\textit{
  Document number 101.
  File version 1.0. 
  This document is from the Standard BMP Methods collection.
    \footnote{For more information and other documents, visit \url{https://www.dbfz.de/en/BMP}. 
    For document version history or to propose changes, visit \url{https://github.com/sashahafner/BMP-methods}.}
}
}

\begin{document}
\maketitle

\section{BMP measurement and validation requirements}
\label{sec:crit}
BMP results that meet \textit{all} the requirements listed in document 100 \citep{BMPdoc100req} can be described as ``validated''.
Otherwise tests should be repeated.
These requirements relate to analysis of inoculum and substrates, test setup, calculations, and finally, validation criteria applied after calculating BMP.
\textbf{Important requirements} include:
\begin{enumerate}
  \item All conditions (blanks, positive controls, substrate) must be replicated in triplicate at the time of data analysis.
  \item Tests continue until daily \ce{CH4} production from individual batches (bottles) during 3 consecutive days is $<$ 1.0\% of the net accumulated volume of methane (substrate batch minus average of blanks). 
\end{enumerate}
After BMP values are calculated, the following \textbf{validation criteria} must be met:
\begin{enumerate}
  \item Mean cellulose BMP is between 340 and 395 NmL\textsubscript{CH\textsubscript{4}} g\textsubscript{VS}\textsuperscript{-1}.
  \item Relative standard deviation for cellulose BMP is no more than 6\%.
\end{enumerate}
\vspace{7mm}

\section{More details}
This document summarizes the minimal requirements for measurement and validation of biochemical methane potential (also called biomethane potential) (BMP) in batch tests.
The present document summarizes only the most important parts of document 100, which is more comprehesive; for all requirements see the document itself \citep{BMPdoc100req}.
Development of the criteria is described in \citet{hafnerImprovingInterlaboratoryReproducibility2020}.
Only results from BMP tests that have followed the requirements given in document 100 \citep{BMPdoc100req} can be considered ``validated''.

\bibliography{bib}

\end{document}

\documentclass[]{article}
\usepackage[version=3]{mhchem}
\usepackage{natbib}
\bibliographystyle{abbrvnat}
\usepackage[colorlinks]{hyperref}
\hypersetup{
  citecolor = {blue},
}
\newcommand{\unit}[1]{\ensuremath{\, \mathrm{#1}}}


\title {Calculation of Methane Production from Gas Density-Based Measurements\footnote{
  Recommended citation: 
Hafner, S.D.; Justesen, C.G.; Thorsen, R.; Astals, S.; Holliger, C.; Koch, K.;, Weinrich, S., Calculation of methane production from gas density-based measurements. Standard BMP Methods document 204, version 1.5. Available online: https://www.dbfz.de/en/BMP (accessed on April 19, 2020).
\newline
  Or see \url{https://www.dbfz.de/en/BMP} for a BibTeX file that can be imported into citation management software.
}
}

\author{Sasha D.Hafner, Camilla G. Justesen, Rasmus Thorsen, \\ Sergi Astals, Christof Holliger, Konrad Koch, \\ and S{\"o}ren Weinrich
\\
\texttt{sasha.hafner@eng.au.dk}\\
}

\date{\today \\
\bigskip
\textit{
  Document number 204.
  File version 1.5. 
  This document is from the Standard BMP Methods collection.
    \footnote{For more information and other documents, visit \url{https://www.dbfz.de/en/BMP}. 
    For document version history or to propose changes, visit \url{https://github.com/sashahafner/BMP-methods}.}
}
}


\begin{document}
\maketitle

\section{Introduction}
Biochemical methane potential (BMP) can be measured using several different methods.
In the gas density BMP (GD-BMP) method, bottle mass loss and vented biogas volume from one or more time intervals are used to determine biogas density, and from that, composition. 
With this information, CH$_4$ production can be calculated from either biogas volume or bottle mass loss.
Development and validation of the method is described in \citet{justesenDevelopmentValidationLowcost2019}.
This document describes calculations for the GD-BMP method and provides an example calculation.

\section{Calculation of methane production}
Using Eq. (\ref{eq:1}), biogas density ($\rho_b$, g/mL) is determined by mass loss ($\Delta m_b$, g) and standardized biogas volume ($V_b$, mL), corrected for water vapor content ($c\textsubscript{H$_2$O}$, g/mL) in the gas. 
\begin{equation}
  \label{eq:1}
  \rho_b=\frac{\Delta m_b}{V_b}-c\textsubscript{H$_2$O}
\end{equation}
Standardized biogas volume is determined from measured vented biogas volume by correcting for moisture, temperature, and pressure, as described in the BMP-methods document 201, on volumetric calculations \citep{BMPdoc201vol}.
Water vapor pressure ($p\textsubscript{H$_2$O}$, kPa) is assumed to be at saturation, and can be calculated using a Magnus-form equation from \citet{alduchovImprovedMagnusForm1996}\footnote{
  Other options exist, and will provide nearly identical values.
}.

\begin{equation}
\label{eq:2_magnus}
   p\textsubscript{H$_2$O} = 0.61094 \cdot e^{\frac{17.625 T_{hs}}{243.04 + T_{hs}}}
\end{equation}

$T_{hs}$ in Eq. (\ref{eq:2_magnus}) indicates the bottle headspace temperature at the time of venting ($^\circ$C). 
The concentration of the water vapor present in the vented biogas ($c\textsubscript{H$_2$O}$) is then calculated from the molar mass of water ($M\textsubscript{H$_2$O}$ = 18.02 g/mol), the partial pressure ($p\textsubscript{H$_2$O}$, kPa), the pressure of biogas in the bottle headspace just prior to venting ($p_{hs}$, kPa), and the molar volume of biogas at standard conditions (here, dry, 101.325 kPa, and 0$^\circ$C).
\begin{equation}
  \label{eq:3}
  c\textsubscript{H$_2$O}=M\textsubscript{H$_2$O} \cdot \frac{p\textsubscript{H$_2$O}}{p_{hs} - p\textsubscript{H$_2$O}} \cdot \frac{1}{v_b}
\end{equation}
The molar volume of biogas ($v_b$) at standard conditions is approximated as 22300 mL/mol \citep{hafnerValidationSimpleGravimetric2015}.

Finally, the mole fraction of \ce{CH4} in biogas ($x\textsubscript{CH$_4$}$, dimensionless) normalized for \ce{CH4} and \ce{CO2} ($x\textsubscript{CH$_4$}$ + $x\textsubscript{CO$_2$}$ = unity) is calculated from the normalized difference in density of \ce{CO2} and biogas\footnote{
  In the original work \citep{justesenDevelopmentValidationLowcost2019} molar mass was calculated as an intermediate step, but this is not needed.
}.
\begin{equation}
  \label{eq:4}
  x\textsubscript{CH$_4$}=\frac{\rho\textsubscript{CO$_2$} - \rho_b}{\rho\textsubscript{CO$_2$}-\rho\textsubscript{CH$_4$}}
\end{equation}
In Eq. (\ref{eq:4}) $\rho\textsubscript{CH$_4$}$ and $\rho\textsubscript{CO$_2$}$ are pure gas densities at standard conditions, which are 0.0007174 and 0.001977 g mL$^{-1}$ for \ce{CH4} and \ce{CO2}, respectively.
From Eq. (\ref{eq:4}), the content of \ce{CH4} in the biogas is known and can be used for calculation of BMP as with gravimetric or volumetric methods \citep{hafnerValidationSimpleGravimetric2015}. 
Eq. (\ref{eq:4}) is based on the assumption that biogas contains only \ce{CH4} and \ce{CO2}.
Over a complete BMP incubation, this is generally a reasonable approximation, but a correction is available \citep{justesenDevelopmentValidationLowcost2019}.

\section{Example calculation} \label{s_example}
In the following example, \ce{CH4} production is calculated from measurements made on a single bottle in a BMP trial.
For a complete BMP trial, the standardized biogas volume \citep{BMPdoc201vol} was 779.2 mL, and the complete total mass loss was 1.070 g.

To find the biogas density ($\rho_b$) with Eq. (\ref{eq:1}), water vapor partial pressure is first calculated using Eq. (\ref{eq:2_magnus}). 
The bottle headspace temperature (\textit{T}$_{hs}$) is assumed to be 30$^\circ$C.
\begin{equation*}
  p\textsubscript{H$_2$O} = 0.61094 \cdot e^{\frac{17.625 \cdot 30\unit{^\circ C}}{243.04 + 30 \unit{^\circ C}}} = 4.237 \unit{kPa}
\end{equation*}
Then, following Eq. (\ref{eq:3}), the concentration of the water vapor ($c\textsubscript{H$_2$O}$) is calculated.
\begin{equation*}
  c\textsubscript{H$_2$O} = 18.016 \unit{g~mol^{-1}} \cdot \frac{4.237 \unit{kPa}}{150 \unit{kPa} - 4.237 \unit{kPa}} \cdot \frac{1 \unit{mol}}{22300 \unit{mL}} = 2.348\cdot10^{-5} \unit{g~mL^{-1}}
\end{equation*}

With $c\textsubscript{H$_2$O}$ and measured biogas volume and bottle mass loss, biogas density can be calculated from Eq. (\ref{eq:1}).
\begin{equation*}
  \rho_b=\frac{1.070 \unit{g}}{779.2 \unit{mL}} - 2.348 \cdot 10^{-5} \unit{g~mL^{-1}} = 1.349 \cdot 10^{-3} \unit{g~mL^{-1}}
\end{equation*}

The mole fraction of CH$_4$ ($x\textsubscript{CH$_4$}$, dimensionless) is calculated from the density of the biogas components using Eq. (\ref{eq:4}). 

\begin{equation*}
  x\textsubscript{CH$_4$}=\frac{1.977  \cdot 10^{-3} \unit{g~mL^{-1}} - 1.349 \cdot 10^{-3} \unit{g~mL^{-1}}}
  {1.977  \cdot 10^{-3} \unit{g~mL^{-1}} - 7.174 \cdot 10^{-4} \unit{g~mL^{-1}}}
  = 0.498
\end{equation*}



\bibliography{bib}

\end{document}

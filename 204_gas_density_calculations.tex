\documentclass[]{article}
\usepackage[version=3]{mhchem}
%\usepackage{natbib}
%\bibliographystyle{abbrvnat}
\usepackage{siunitx}
\usepackage[colorlinks]{hyperref}

\newcommand{\unit}[1]{\ensuremath{\, \mathrm{#1}}}


\title {Calculation of Methane Production from Gas Density-Based Measurements\footnote{
  Recommended citation: 
Hafner, S.D.; Justesen, C.G.; Thorsen, R.; Astals, S.; Holliger, C.; Koch, K.;, Weinrich, S., Calculation of methane production from gas density-based measurements. Standard BMP Methods document 204, version 1.5. Available online: https://www.dbfz.de/en/BMP (accessed on April 19, 2020).
\newline
  Or see \url{https://www.dbfz.de/en/BMP} for a BibTeX file that can be imported into citation management software.
}
}

\author{Sasha D.Hafner, Camilla G. Justesen, \\ Rasmus Thorsen, Sergi Astals, \\ Christof Holliger, Konrad Koch, \\ and S{\"o}ren Weinrich
\\
\texttt{sasha.hafner@eng.au.dk}\\
}

\date{\today \\
\bigskip
\textit{
  Document number 204.
  File version 1.5. 
  This document is from the Standard BMP Methods collection.
    \footnote{For more information and other documents, visit \url{https://www.dbfz.de/en/BMP}. 
    For document version history or to propose changes, visit \url{https://github.com/sashahafner/BMP-methods}.}
}
}


\begin{document}
\maketitle

\section{Introduction}
In the gas density BMP (GD-BMP) method, bottle mass loss and vented biogas volume from one or more time intervals are used to determine biogas density, and from that, composition. 
With this information, CH$_4$ production can be calculated from either biogas volume or bottle mass loss.
This document describes calculations for the GD-BMP method and provides an example calculation.

\section{Calculation of methane production}
Using Eq. (\ref{eq:1}), biogas density ($d_b$, g/mL) is determined by mass loss ($\Delta m_b$, g) and standardized biogas volume ($V_b$, mL), corrected for water vapor content ($m_{H_2O}$, g/mL) in the gas. 
\begin{equation}
  \label{eq:1}
  d_b=\frac{\Delta m_b}{V_b}-m_{H_2O}
\end{equation}
Standardized biogas volume is determined from measured vented biogas volume by correcting for moisture, temperature, and pressure, as described in the BMP-methods document on volumetric calculations (Hafner, 2019).
Water vapor pressure ($P_{H_2O}$, kPa) is assumed to be at saturation, and can be calculated using a Magnus-form equation from Alduchov and Eskridge (1996)\footnote{
  Other options exist, and will provide nearly identical values.
}.

\begin{equation}
\label{eq:2_magnus}
   P_{H_2O} = 0.61094 \cdot e^{\frac{17.625 T}{243.04 + T}}
\end{equation}

$T_{hs}$ in Eq. \ref{eq:2_magnus} indicates the bottle headspace temperature at the time of venting ($^\circ$C). 
The mass of the water vapor present in the vented biogas ($m_{H_2O}$) is then calculated from the molar mass of water ($M_{H_2O}$ = 18.02 g/mol), the partial pressure ($P_{H_2O}$, kPa), the pressure of biogas in the bottle headspace just prior to venting ($P_{hs}$, kPa), and the molar volume of biogas at standard conditions (here, dry, 101.325 kPa, and 0$^\circ$C).
\begin{equation}
  \label{eq:3}
  m_{H_2O}=M_{H_2O} \cdot \frac{P_{H_2O}}{P_{hs}-P_{H_2O}} \cdot \frac{1}{v_b}
\end{equation}
The molar volume of biogas ($v_b$) at standard conditions is approximated as 22300 mL/mol (Hafner et al., 2015) and the biogas pressure just prior to venting was assumed to be 150 kPa.

The molar mass of biogas ($M_b$, g/mol) is then obtained from the density and molar volume of the biogas.
\begin{equation}
  \label{eq:4}
  M_b=d_b \cdot v_b
\end{equation}
Finally, the mole fraction of \ce{CH4} in biogas ($x_{CH_4}$, dimensionless) normalized for \ce{CH4} and \ce{CO2} ($x_{CH_4}$ + $x_{CO_2}$ = unity) is calculated from the normalized difference in molar mass of \ce{CO2} and biogas.
\begin{equation}
  \label{eq:5}
  x_{CH_4}=\frac{M_{CO_2}-M_b}{M_{CO_2}-M_{CH_4}}
\end{equation}
From Eq. \ref{eq:5}, the content of \ce{CH4} in the biogas is known and can be used for calculation of BMP as with gravimetric or volumetric methods (Hafner et al., 2015). 
Eq. \ref{eq:5} is based on the assumption that biogas contains only \ce{CH4} and \ce{CO2}.
%Flushing gas will affect the results, but could be accounted for in calculations (Hafner et al., 2015).

\section{Example calculation} \label{s_example}
In the following example, \ce{CH4} production is calculated from measurements made on a single bottle in a BMP trial.
For a complete BMP trial, the standardized biogas volume was 779.2 mL, and the complete total mass loss was 1.070 g.

To find the biogas density ($d_b$) with equation \ref{eq:1}, water vapor partial pressure is first calculated using Eq. (\ref{eq:2_magnus}). 
The bottle headspace temperature (\textit{T}$_{hs}$) was assumed to be 30$^\circ$C.
\begin{equation*}
   P_{H_2O} = 0.61094 \cdot e^{\frac{17.625 \cdot 30^\circ C}{243.04 + 30 ^\circ C}} = 4.237\ kPa
\end{equation*}
Then, following equation \ref{eq:3}, the mass of the water vapor ($m_{H_2O}$) is calculated.
\begin{equation*}
  m_{H_2O} = \SI{18.016} {g/mol} \cdot \frac{\SI{4.237}{kPa}}{\SI{150}{kPa} - \SI{4.237}{kPa}} \cdot \frac{\SI{1}{mol}}{\SI{22300}{mL}} = \SI{2.348e-5}{mg/mL}
\end{equation*}

With $m_{H_2O}$ and measured biogas volume and bottle mass loss, biogas density can be calculated from Eq. (\ref{eq:1}).
\begin{equation*}
  d_b=\frac{1.070\ g}{779.2\ mL} - 2.348 \cdot 10^{-5} \frac{g}{mL} = 1.35 \cdot 10^{-3} \frac{g}{mL}
\end{equation*}
The molar mass of biogas (M$_b$, [g/mol]) is obtained from the density and molar volume of the biogas (eq. \ref{eq:4}).

\begin{equation*}
  \centering
  M_b= 1.35 \cdot 10^{-3} \frac{g}{mL} \cdot 22300\ \frac{mL}{mol} = 30.11\ \frac{g}{mol}
\end{equation*}
The mole fraction of CH$_4$ (x$_{CH_4}$, dimensionless) is calculated from the molar masses of the biogas components using Eq. (\ref{eq:5}). 

\begin{equation*}
  x_{CH_4}=\frac{44.01\ \frac{g}{mol}-30.11\ \frac{g}{mol}}{44.01\ \frac{g}{mol}-16.042\ \frac{g}{mol}} = 0.497
\end{equation*}


\begin{thebibliography}{1}

\bibitem{bmpmethods}
Hafner, S.D.,
    \newblock{2019},
    \newblock{Calculation of methane production from volumetric measurements, part of the BMP-methods repository},
    \newblock{\url{https://github.com/sashahafner/BMP-methods}}

\bibitem{magnus}
Alduchov, O.A., Eskridge, R.E.,   
    \newblock{1996},
    \newblock{Improved Magnus form approximation of saturation vapor pressure.}, 
    \newblock{Journal of Applied Meteorology} 35: 601-609

\bibitem{validation}
Hafner, S.D., Rennuit, C., Triolo, J.M., Richards, B.K.,
    \newblock{2015},
    \newblock{Validation of a simple gravimetric method for measuring biogas production in laboratory experiments.},
        \newblock{Biomass and Bioenergy} 83: 297-301

\end{thebibliography}

\end{document}

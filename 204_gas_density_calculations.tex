\documentclass[]{article}
\usepackage[version=3]{mhchem}
\usepackage{natbib}
\bibliographystyle{abbrvnat}
\usepackage[colorlinks]{hyperref}
\hypersetup{
  citecolor = {blue},
}
\newcommand{\unit}[1]{\ensuremath{\, \mathrm{#1}}}


\title {Calculation of Methane Production from Gas Density-Based Measurements\footnote{
  Recommended citation: 
Hafner, S.D.; Justesen, C.G.; Thorsen, R.; Astals, S.; Holliger, C.; Koch, K.;, Weinrich, S., Calculation of methane production from gas density-based measurements. Standard BMP Methods document 204, version 1.7. Available online: https://www.dbfz.de/en/BMP (accessed on September 7, 2020).
\newline
  Or see \url{https://www.dbfz.de/en/BMP} for a BibTeX file that can be imported into citation management software.
}
}

\author{Sasha D.Hafner, Camilla G. Justesen, Rasmus Thorsen, \\ Sergi Astals, Christof Holliger, Konrad Koch, \\ and S{\"o}ren Weinrich
}

\date{\today \\
\bigskip
\textit{
  Document number 204.
  File version 1.7. 
  This document is from the Standard BMP Methods collection.
    \footnote{For more information and other documents, visit \url{https://www.dbfz.de/en/BMP}. 
    For document version history or to propose changes, visit \url{https://github.com/sashahafner/BMP-methods}.}
}
}


\begin{document}
\maketitle

\section{Introduction}
Biochemical methane potential (BMP) can be measured using several different methods.
In the gas density BMP (GD-BMP) method, bottle mass loss and vented biogas volume from one or more time intervals are used to determine biogas density, and from that, composition. 
With this information, CH$_4$ production can be calculated from either biogas volume or bottle mass loss.
Development and validation of the method is described in \citet{justesenDevelopmentValidationLowcost2019}.
This document describes calculations for the GD-BMP method and provides an example calculation.
Note that all these calculations can be carried out easily using the web application OBA (\url{https://biotransformers.shinyapps.io/oba1/}) or the biogas package (\url{https://CRAN.R-project.org/package=biogas}) \citep{hafnerSoftwareBiogasResearch2018}. 
For a detailed measurement protocol, see document 304 from the Standard BMP Methods website \citep{BMPdoc304gasdens}.

\section{GD-BMP algorithms}
\label{sec:algs}
There are two algorithms that can be used for GD-BMP calculations, which differ only in the way mass loss is calculated.
The general and recommended approach is referred to as GD$_t$ (t for total).
For GD$_t$, mass loss for each individual bottle ($\Delta m_b$) is taken as the sum of mass loss across all individual incubation intervals, which is exactly identical to the difference between initial and final mass.
The alternative is referred to as GD$_v$, and should be used only when bottle leakage is significant.
With GD$_v$, $\Delta m_b$ is taken as the sum of the difference between pre- and post-venting mass over the entire BMP test. 
For both GD$_t$ and GD$_v$, biogas volume $V_b$ is taken as the sum of standardized volume across all measurement intervals.

These algorithms are described in detail in \citet{justesenDevelopmentValidationLowcost2019} (see Fig. 2, Section 2.1.3, and Table S1).

\section{Caclulation of bottle leakage}
For an individual bottle over a single incubation interval, bottle leakage is calculated as the difference between the pre- and post-venting bottle mass.\footnote{
  For details on making these measurements see document 304 \citep{BMPdoc304gasdens}.
}
The sum of these values across all measurement intervals is an estimate of total leakage for an individual bottle.

Leakage detection limit can be estimated from the standard deviation in measured mass of sealed bottles with water only (``water controls'') over all measurements, $s_{wc}$. 
From this value a standard deviation in the sum of $k$ values can be determined, and the limit of detection ($LOD$) may be taken as 3 (or more) times this value.
\begin{equation}
  \label{eq:lod}
  LOD = 3 s_{wc} \sqrt{k}
\end{equation}

In Eq. (\ref{eq:lod}) $k$ is the total number of measurement intervals, and $s_{wc}$ is the standard deviation in the measured mass of a water control bottle over all measurement intervals. 
With two or more water control bottles, $s_{wc}$ can be taken as the mean of values calculated separately for each bottle.
Measured leakage larger than the detection limit is strong evidence that leakage occured. 
In this case, the GD$_v$ algorithm may be used (Section \ref{sec:algs}).
However, precision is higher with GD$_t$ and a more prudent approach is to repeat the test taking care to avoid leakage.
For large leaks (e.g., perhaps $>$ 20\% of total mass loss) the test should always be repeated.

More details on leakage calculations can be found in \citet{hafnerQuantificationLeakageBatch2018}. 

\section{Calculation of methane production}
Using Eq. (\ref{eq:1}), biogas density ($\rho_b$, g/mL) is determined by mass loss ($\Delta m_b$, g) and standardized biogas volume ($V_b$, mL), corrected for water vapor content ($c\textsubscript{H$_2$O}$, g/mL) in the gas. 
\begin{equation}
  \label{eq:1}
  \rho_b=\frac{\Delta m_b}{V_b}-c\textsubscript{H$_2$O}
\end{equation}
Standardized biogas volume is determined from measured vented biogas volume by correcting for moisture, temperature, and pressure, as described in the BMP-methods document 201, on volumetric calculations \citep{BMPdoc201vol}.
Water vapor pressure ($p\textsubscript{H$_2$O}$, kPa) is assumed to be at saturation, and can be calculated using a Magnus-form equation from \citet{alduchovImprovedMagnusForm1996}\footnote{
  Other options exist, and will provide nearly identical values.
}.

\begin{equation}
\label{eq:2_magnus}
   p\textsubscript{H$_2$O} = 0.61094 \cdot e^{\frac{17.625 T_{hs}}{243.04 + T_{hs}}}
\end{equation}

$T_{hs}$ in Eq. (\ref{eq:2_magnus}) indicates the bottle headspace temperature at the time of venting ($^\circ$C). 
The concentration of the water vapor present in the vented biogas ($c\textsubscript{H$_2$O}$) is then calculated from the molar mass of water ($M\textsubscript{H$_2$O}$ = 18.02 g/mol), the partial pressure ($p\textsubscript{H$_2$O}$, kPa), the pressure of biogas in the bottle headspace just prior to venting ($p_{hs}$, kPa), and the molar volume of biogas at standard conditions (here, dry, 101.325 kPa, and 0$^\circ$C).
\begin{equation}
  \label{eq:3}
  c\textsubscript{H$_2$O}=M\textsubscript{H$_2$O} \cdot \frac{p\textsubscript{H$_2$O}}{p_{hs} - p\textsubscript{H$_2$O}} \cdot \frac{1}{v_b}
\end{equation}
The molar volume of biogas ($v_b$) at standard conditions is approximated as 22300 mL/mol \citep{hafnerValidationSimpleGravimetric2015}.

Finally, the mole fraction of \ce{CH4} in biogas ($x\textsubscript{CH$_4$}$, dimensionless) normalized for \ce{CH4} and \ce{CO2} ($x\textsubscript{CH$_4$}$ + $x\textsubscript{CO$_2$}$ = unity) is calculated from the normalized difference in density of \ce{CO2} and biogas\footnote{
  In the original work \citep{justesenDevelopmentValidationLowcost2019} molar mass was calculated as an intermediate step, but this is not needed.
}.
\begin{equation}
  \label{eq:4}
  x\textsubscript{CH$_4$}=\frac{\rho\textsubscript{CO$_2$} - \rho_b}{\rho\textsubscript{CO$_2$}-\rho\textsubscript{CH$_4$}}
\end{equation}
In Eq. (\ref{eq:4}) $\rho\textsubscript{CH$_4$}$ and $\rho\textsubscript{CO$_2$}$ are pure gas densities at standard conditions, which are 0.0007174 and 0.001977 g mL$^{-1}$ for \ce{CH4} and \ce{CO2}, respectively.
From Eq. (\ref{eq:4}), the content of \ce{CH4} in the biogas is known and can be used for calculation of BMP as with gravimetric or volumetric methods \citep{hafnerValidationSimpleGravimetric2015}. 
Eq. (\ref{eq:4}) is based on the assumption that biogas contains only \ce{CH4} and \ce{CO2}.
Over a complete BMP incubation, this is generally a reasonable approximation, but a correction is available \citep{justesenDevelopmentValidationLowcost2019}.

\section{Example calculation} \label{s_example}
In the following example, \ce{CH4} production is calculated from measurements made on a single bottle in a BMP trial.
For a complete BMP trial, the standardized biogas volume\footnote{
  See document 201 \citep{BMPdoc201vol} for information on calculation of standardized biogas volume.
} was 779.2 mL, and the complete total mass loss was 1.070 g.
Total leakage mass loss was only 0.003 g, much smaller than the detection limit of 0.04 g, so the GD$_t$ approach is appropriate.
Data are from \citet{justesenDevelopmentValidationLowcost2019} (Experiment 2, cellulose bottle).

To find the biogas density ($\rho_b$) with Eq. (\ref{eq:1}), water vapor partial pressure is first calculated using Eq. (\ref{eq:2_magnus}). 
The bottle headspace temperature (\textit{T}$_{hs}$) is assumed to be 30$^\circ$C.
\begin{equation*}
  p\textsubscript{H$_2$O} = 0.61094 \cdot e^{\frac{17.625 \cdot 30\unit{^\circ C}}{243.04 + 30 \unit{^\circ C}}} = 4.237 \unit{kPa}
\end{equation*}
Then, following Eq. (\ref{eq:3}), the concentration of the water vapor ($c\textsubscript{H$_2$O}$) is calculated.
\begin{equation*}
  c\textsubscript{H$_2$O} = 18.016 \unit{g~mol^{-1}} \cdot \frac{4.237 \unit{kPa}}{150 \unit{kPa} - 4.237 \unit{kPa}} \cdot \frac{1 \unit{mol}}{22300 \unit{mL}} = 2.348\cdot10^{-5} \unit{g~mL^{-1}}
\end{equation*}

With $c\textsubscript{H$_2$O}$ and measured biogas volume and bottle mass loss, biogas density can be calculated from Eq. (\ref{eq:1}).
\begin{equation*}
  \rho_b=\frac{1.070 \unit{g}}{779.2 \unit{mL}} - 2.348 \cdot 10^{-5} \unit{g~mL^{-1}} = 1.349 \cdot 10^{-3} \unit{g~mL^{-1}}
\end{equation*}

The mole fraction of CH$_4$ ($x\textsubscript{CH$_4$}$, dimensionless) is calculated from the density of the biogas components using Eq. (\ref{eq:4}). 

\begin{equation*}
  x\textsubscript{CH$_4$}=\frac{1.977  \cdot 10^{-3} \unit{g~mL^{-1}} - 1.349 \cdot 10^{-3} \unit{g~mL^{-1}}}
  {1.977  \cdot 10^{-3} \unit{g~mL^{-1}} - 7.174 \cdot 10^{-4} \unit{g~mL^{-1}}}
  = 0.498
\end{equation*}

\bibliography{bib}

\end{document}




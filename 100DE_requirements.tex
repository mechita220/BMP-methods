\documentclass[]{article}
\usepackage[version=3]{mhchem}
\usepackage{natbib}
\bibliographystyle{abbrvnat}
\usepackage[colorlinks]{hyperref}
\hypersetup{
  citecolor = {blue},
}
\newcommand{\unit}[1]{\ensuremath{\, \mathrm{#1}}}

\title {Anforderungen an die Messung und Validierung des biochemischen Methanpotenzials (BMP)\footnote{
  Empfohle Zitierung: 
Holliger, C.; Fruteau de Laclos, H.; Hafner, S.D.; Koch, K.; Weinrich, S.; Astals, S. et al. Requirements for measurement and validation of biochemical methane potential (BMP). Standard BMP Methods document 100, version 1.8. Online verfügbar unter https://www.dbfz.de/en/BMP (abgerufen am TT.MM.JJJJ).
\newline
  Unter \url{https://www.dbfz.de/en/BMP} gibt es eine BibTeX-Datei, die in alle gängigen Literaturverwaltungsprogramme importiert werden kann.
}}
\author{
Christof Holliger, 
H{\'e}l{\`e}ne Fruteau de Laclos,
Sasha D. Hafner,\\
Konrad Koch,
S{\"o}ren Weinrich,
Sergi Astals, \\
Madalena Alves, 
Diana Andrade,
Irini Angelidaki,\\
Lise Appels,
Samet Azman,
Alexandre Bagnoud \\
Urs Baier,
Yadira Bajon Fernandez,
Jan Bartacek,\\
Federico Battista,
David Bolzonella,
Claire Bougrier,\\
Camilla Braguglia,
Pierre Buffi{\`e}re,
Marta Carballa,\\
Arianna Catenacci,
Vasilis Dandikas,
Fabian de Wilde,\\
Sylvanus Ekwe,
Elena Ficara,
Ioannis Fotidis,\\
Jean-Claude Frigon,
Agata Gallipoli,
J{\"o}rn Heerenklage,\\
Pavel Jenicek,
Judith Krautwald,
Ralph Lindeboom,\\
Jing Liu,
Javier Lizasoain,
Rosa Marchetti,\\
Florian Monlau,
Mihaela Nistor,
Hans Oechsner,\\
Jo{\~a}o V{\'i}tor Oliveira,
Andr{\'e} Pauss,
S{\'e}bastien Pommier,\\
Francisco Raposo,
Thierry Ribeiro,
Christian Schaum,\\
Els Schuman,
Sebastian Schwede,
Mariangela Soldano,\\
Anton Taboada,
Michel Torrijos,
Miriam van Eekert,\\
Jules van Lier, 
Isabella Wierinck
} 

\date{\today \\
\bigskip
\textit{
  Document 100.
  Version 1.8 (Deutsche Version).
  Dieses Dokument ist Teil der Methodensammlung zur Standardisierung von Batchversuchen.\footnote{ Weitere Informationen und andere Dokumente finden Sie unter \url{https://www.dbfz.de/en/BMP}. 
    Alle Dokumentversionen gibt es unter \url{https://github.com/sashahafner/BMP-methods}; dort ist es auch möglich, Änderungen vorzuschlagen.}  
}
}

\begin{document}
\maketitle

\section{Einleitung}
Dieses Dokument stellt die Mindestanforderungen für die Messung und Validierung des biochemischen Methanpotenzials (auch Biomethanpotenzial genannt, BMP) in Batchversuchen vor und entspricht dem Konsens von mehr als 40 Biogas-Experten. Die Liste der Anforderungen basiert auf dem Leitfaden von  \citet{holligerStandardizationBiomethanePotential2016}, mit Ausnahme von einigen Anpassungen der Validierungskriterien gemäß der Erkenntnisse aus  \citet{hafnerImprovingInterlaboratoryReproducibility2020} und zusätzlichen Details zur standardisierten Berechnung des BMP.
Details und weiterführenden Empfehlungen können den entsprechenden Veröffentlichungen entnommen werden \citep{holligerStandardizationBiomethanePotential2016,hafnerImprovingInterlaboratoryReproducibility2020}.

\section{Anforderungen an die Messung des BMP}
\label{sec:req}
\subsection{Analyse von Substrat und Inokulum}
\label{sec:analysis}
  Die Bestimmung des Gehaltes an organischer Trockensubstanz (oTS) sowohl des Substrates als auch des Inokulums ist erforderlich, um die jeweiligen Zugabemengen für ein gewähltes Inokulum-Substrat-Verhältnis (ISV) zu ermitteln. Darüber hinaus wird es für die Berechnung des Biomethanpotentials des Substrates benötigt. Eine detaillierte Beschreibung der Bestimmung der TS und oTS ist u.a. in \citet{strachDeterminationTotalSolidsDE2016} und \citet{bairdStandardMethodsExamination2017} zu finden, sowie in einem ausführlichen kostenlosen Dokument der US-EPA \citep{epaMethod1684Total2001} (in Englisch). 

  \begin{enumerate}
    \item Trockensubstanz (TS). Dreifach-Bestimmung für das Inokulum und alle Substrate durch Trocknung bei 105$^\circ$C bis zur Gewichtskonstanz.
    \item Organische Trockensubstanz (oTS). Dreifach-Bestimmung für das Inokulum und alle Substrate durch Veraschung der getrockneten Probe bei 550$^\circ$C bis zur Gewichtskonstanz.
    \end{enumerate}

\subsection{Versuchsaufbau und -dauer}
\label{sec:setup}
\begin{enumerate}
  \item Proben. 
    Alle Batchversuche müssen neben dem eigentlichen Ansatz mit Inokulum und Substrat zusätzliche folgende Ansätze enthalten: Einen Ansatz mit ausschließlich Inokulum (Negativkontrolle bzw. Blank) sowie einen Ansatz mit Inokulum und mikrokristalliner Cellulose als Positivkontrolle.\footnote{
      Andere Substrate zur Positivkontrolle sind theoretisch ebenfalls möglich \citep{kochEvaluationCommonSupermarket2020}, allerdings wurden nur mit Cellulose umfangreiche Tests durchgeführt, die zur Entwicklung der in Abschnitt \ref{sec:crit} beschriebenen Validierungskriterien geführt haben
      \citep{hafnerImprovingInterlaboratoryReproducibility2020}.
    }
    \item Replikate. 
    Jeder der oben beschriebenen Versuchsansätze ist mindestens als Dreifachbestimmung anzusetzen.\footnote{
      Kommt es während des Versuches zum Verlust einer Flasche, beispielsweise durch Beschädigung, sodass für einen der Ansätze nur noch 2 Flaschen für die Auswertung zur Verfügung stehen, kann der gesamte Versuch nicht validiert werden (vgl. Abschnitt \ref{sec:crit}).
      Daher ist es ratsam, besser 4 (oder mehr) Replikate anzusetzen, insbesondere bei den Blanks.
      Ausreißer können eliminiert werden, wenn ein begründeter Verdacht auf einen Messfehler besteht (z.B. Undichtigkeit), aber die verbleibende Anzahl von Replikaten für die Auswertung muss mindestens 3 betragen.
    }
    Die Mindestanzahl der in einem Batchtest mit einem Substrat anzusetzenden Flaschen beträgt daher 9 (3 x Blanks, 3 x Cellulose, 3 x Substrat).
  \item Dauer. 
    Der BMP-Test darf erst beendet werden, wenn die tägliche \ce{CH4}-Produktion aus den einzelnen Ansätzen während 3 aufeinanderfolgender Tagen weniger als 1\% des akkumulierten Nettovolumens an Methan aus dem Substrat (Substratansatz minus Durchschnitt der Blanks) beträgt.
    Bei manuellen oder anderen Messmethoden, bei denen Messungen nicht jeden Tag durchgeführt werden, kann der Abbruch am Ende des ersten Messintervalls von mindestens 3 Tagen erfolgen, an der die Produktionsrate unter das Abbruchkriterium von 1\% fällt (oder zwei oder mehr Intervalle, die sich auf mindestens 3 Tage summieren und alle Raten unter 1\% aufweisen).
    Wenn verschiedene Substrate parallel getestet werden, soll der jeweilige Substratansatz erst dann beendet werden, wenn auch das letzte Replikat das Abbruchkriterium erreicht hat.
    Blanks dürfen erst beendet werden, wenn alle Flaschen des gesamten Ansatzes (Substrat und Positivkontrolle) das Abbruchkriterium unterschreiten.
    Die Fortsetzung der Tests über dieses 1\%-Abbruchkriterium hinaus ist legitim und kann dazu beitragen, sicherzustellen, dass die in Abschnitt \ref{sec:crit} aufgeführten Validierungskriterien erfüllt werden (beispielsweise bezüglich des erforderlichen Bereiches des Biomethanpotentials der Cellulose).
\end{enumerate}

\section{Berechnungen}
\label{sec:calculations}
\begin{enumerate}
  \item Datenaufbereitung.
    Das genormte \ce{CH4}-Volumen (trocken, 0$^\circ$C, 101,325 kPa) wird aus den Rohdaten nach standardisierten Methoden berechnet.\footnote{
      Eine detaillierte Beschreibungen der Berechnungen sind für die folgenden Messmethoden der Methodensammlung zur Standardisierung von Batchversuche enthalten (\url{https://www.dbfz.de/en/BMP}): volumetrisch (Document 201) \citep{BMPdoc201vol}, manometrisch (Document 202) \citep{BMPdoc202man}, gravimetrisch (Document 203) \citep{BMPdoc203grav}, und mittels Gasdichte (Document 204) \citep{BMPdoc204gasdens}.
    }
  \item Einheit.
	  Das Biomethanpotential wird als genormtes \ce{CH4}-Volumen (trocken, 0$^\circ$C, 101,325 kPa; häufig als "Normvolumen" bezeichnet) bezogen auf die Masse an zugegebener organischer Trockensubstanz (oTS) mit der Einheit NmL\textsubscript{CH\textsubscript{4}} g\textsubscript{oTS}\textsuperscript{-1} angegeben. Insbesondere bei eher flüssigen Substraten wird mitunter auch der chemische Sauerstoffbedarf (CSB) als Bezugsgröße gewählt (NmL\textsubscript{CH\textsubscript{4}} g\textsubscript{CSB}\textsuperscript{-1}).
  \item Berechnung des Biomethanpotentials.
    Das Biomethanpotential aller Substrate (inklusive der Positivkontrolle) errechnet sich durch den Abzug der \ce{CH4}-Produktion der Blanks (im Durchschnitt aller Flaschen) von der \ce{CH4}-Produktion der Substrate mit Inokulum und der Normierung auf die zugeführte Menge an oTS.
    Mögliche Differenzen in den Einwaagen von Substrat oder Inokulum in den einzelnen Ansätzen müssen bei der Berechnung unbedingt berücksichtigt werden.
    Die Berechnungen solltem einem standardisierten Ansatz folgen.\footnote{
      Die Berechnung des Biomethanpotential ist ausführlich in Dokument 200 der Methodensammlung zur Standardisierung von Batchversuche beschrieben (\url{https://www.dbfz.de/en/BMP}).
    }
    \item Berechnung der Standardabweichung.
    Die gemeinsam mit dem ermittelten durchschnittlichen Biomethanpotential anzugebende Standardabweichung ($n \ge 3$) muss die Varianz sowohl innerhalb des Substratansatzes, als auch die der Blanks berücksichtigen. Zudem sollte die Messunsicherheit bei der Bestimmung des oTS und die der Einwaagemengen ebenfalls Berücksichtigung finden.\footnote{
      Weiterführende Informationen diesbezüglich in Dokument 200. 
    }
\end{enumerate}

\section{Validierungskriterien}
\label{sec:crit}
Nur Versuche, in denen \textit{alle} der folgenden Kriterien erfüllt wurden, gelten als "validiert".\footnote{
Die Validierungskriterien sind auch in Dokument 101 zu finden \citep{BMPdoc101val}, in dem diese zentralen Kriterien in übersichtlicher Form zusammengefasst sind.
}
Andernfalls kann der gesamte Versuch nicht validiert werden und muss entsprechend wiederholt werden.

\begin{enumerate}
  \item Alle erforderlichen Kriterien zur Bestimmung des Biomethanpotentials, wie diese in Abschnitt \ref{sec:req} beschrieben sind, wurden eingehalten (inklusive der notwendigen Dauer des Versuches). Alle Berechnungen erfolgten gemäß der Vorgaben in Abschnitt \ref{sec:calculations}.
  \item Das durschnittliche Biomethanpotential für mikrokristalline Cellulose liegt zwischen 340 and 395 NmL\textsubscript{CH\textsubscript{4}} g\textsubscript{oTS}\textsuperscript{-1}.
  \item Die relative Standardabweichung des Biomethanpotentials (absolute Standardabweichung geteilt durch das mittlere Methanpotential) der Cellulose (inklusive der Varianz der Blanks und der Messungenauigkeit bei der oTS-Bestimmung) ist kleiner als 6\%.
\end{enumerate}

\bibliography{bib}

\end{document}
